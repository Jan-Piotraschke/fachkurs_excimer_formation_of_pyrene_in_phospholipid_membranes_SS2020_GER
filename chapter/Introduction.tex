\pagestyle{headings}
\pagenumbering{arabic} 
\setcounter{page}{1}

\section{Einleitung}
NOTE: Einleitung länger machen -> 1-2 Seiten! \\\\

Der vorliegende Versuch untersucht die konzentrations- bzw. temperaturabhängige Excimerbildung von Pyren in Ei-PC (Eilecithin) und DPPC (Dipalmitoylphosphatidylcholin) Vesikeln.\\
Anschließend werden die verschiedenen Lateraldiffusionskoeffizienten von Pyren errechnet. \\

NOTE: Pyren (seine Besonderheiten: etc.etc. ) ermölgicht eine Excimerenbildung basierend auf den Grundalgen ... ...
\\\\
Der Versuch benutzt die Methode der Fluoreszenzmessung. Fluoreszenz kann i.a. bei einem delokalisierten $\pi$-System auftreten, indem ein Chromophor durch die elektrische Komponente einer elektromagnetischen Strahlung vom $S_0$ Grundenergiezustand ausgehend angeregt wird und nach einer gewissen Zeit der $S_1$ Zustand sich mit Strahlung wieder auf den Grundzustand relaxiert.
