\pagestyle{headings}
\pagenumbering{arabic} 
\setcounter{page}{1}

\section{Einleitung}
NOTE: Einleitung länger machen -> 1-2 Seiten! \\\\

Der vorliegende Versuch untersucht die konzentrations- bzw. temperaturabhängige Excimerbildung von Pyren in Ei-PC (Eilecithin) und DPPC (Dipalmitoylphosphatidylcholin) Lipidvesikeln.\\
Anschließend werden die verschiedenen Lateraldiffusionskoeffizienten von Pyren errechnet. \\

NOTE: Pyren (seine Besonderheiten: etc.etc. ) ermölgicht eine Excimerenbildung basierend auf den Grundalgen ... ...
\\\\
Der Versuch benutzt die Methode der Fluoreszenzmessung anhand des Fluorophors Pyren. Fluoreszenz kann i.a. bei einem delokalisierten $\pi$-System auftreten, indem ein Chromophor durch die elektrische Komponente einer elektromagnetischen Strahlung vom $S_0$ Grundenergiezustand ausgehend angeregt wird und nach einer gewissen Zeit der $S_1$ Zustand sich mit Abgabe von Strahlung und Wärme wieder auf den Grundzustand relaxiert.\\
Pyren ist in der Mebranspektroskopie ein häufig verwendeter Fluorophore. Angeregte Monomere von Pyren können durch Kollisionswechselwirkung mit einem nicht angeregten Pyrenmonomer ein angeregtes Dimer bilden, welches Excimer genannt wird. Jenes Excimer besitzt eine geringere Energie,
weshalb seine Fluoreszenzemission im Vergleich zum angeregten Monomer zu höhreren Wellenlängen verschoben ist. \\ Die Bildung des Excimers ist abhängig von der Pyrenkonzentration innerhalb der Lipidmembran und der Beweglichkeit der Membran. Dadurch kann die Lateraldiffusion innerhalb der Membran analysiert werden, indem man .... .... \\\\\\

Es wird erwartet, dass bei höherer Temperatur auch ein größeres Excimersignal auftritt, da angeregte Monomere durch die erhöhte 
Membranbeweglichkeit öfters Kollisionen mit unangeregten Monomeren hat.